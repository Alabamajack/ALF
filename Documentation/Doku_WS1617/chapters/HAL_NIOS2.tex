\chapter{NIOS2 - Treiber / Hardware Abstraction}
Hier wird kurz der Aufbau der HAL bzw. der Aufbau der jetzt zur Verfügung stehenden Treiber des NIOS2 Cores erläutert. Genauere Dokumentation ist in der Doxygen Dokumentation zu finden, die im Anhang mitgeliefert wird. 
\section{Display}
Das Display kann sogar Zahlen anzeigen
\section{Motor}
Die Klasse Drive ist für die Ansteuerung des Motors zuständig. Dabei ist es möglich die maximale Geschwindigkeit im Bereich zwischen 0\% und 100\% zu begrenzen. Eine weitere Funktion ermöglicht das Setzen der Richtung und der Geschwindigkeit, das dann in ein PWM Signal für den Motor umgerechnet wird. 
\section{Lenkung}
Die Lenkung wird mittels der Klasse Steering ermöglicht. Eine Init() Funktion setzt denn maximalen Lenkwinkel, der für die Lenkung benutzt wird. Eine weitere Funktion Set() setzt dann den tatsächlichen Winkel.
\section{MPU6050}
Das mpu6050 Modul ist über den IIC Bus angeschlossen und kann die 3 Beschleunigungsachsen, 3 Drehachsen und die Temperatur auslesen.
\section{Ultraschall}
Die vier zur Verfügung stehenden Ultraschallsensoren sind über den gleichen IIC Bus, wie das MPU6050 Modul angebunden.  

