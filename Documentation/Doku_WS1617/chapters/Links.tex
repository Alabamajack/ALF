\chapter{Hilfreiche Verweise}
Nachfolgend werden einige Links inklusiver kurzer Beschreibung aufgezählt, die während der Projekts eine hilfreiche Anlaufstelle für Problemlösungen waren.
\begin{itemize}
	\item \href{https://eewiki.net/display/linuxonarm/DE0-Nano-SoC+Kit}{Ubuntu on DE0} - Das im Projekt verwendete Tutorial um eine Ubuntu Distribution auf dem \ac{SoC} lauffähig zu machen.
	\item \href{https://www.altera.com/support/support-resources/design-examples/intellectual-property/embedded/nios-ii/exm-multi-nios2-hardware.html}{Multicore} - Ein Multicore Beispiel für mehrere NIOS2 Prozessoren und die Inter Prozess Kommunikation
	\item \href{https://rocketboards.org/foswiki/view/Documentation/BuildingMultiProcessorSystems}{Multiprozessor I} - Ein Beispiel wie man ein Multicoresystem für den Cyclone V mit den \ac{HPS} und mindestens einem NIOS2 aufbauen kann.
	\item \href{https://rocketboards.org/foswiki/view/Documentation/DatamoverDesignExample}{Multiprozessor II} - Ein Beispiel für die Kommunikation zwischen mehreren Prozessoren.
	\item \href{http://elinux.org/images/f/f9/Petazzoni-device-tree-dummies_0.pdf}{Device Tree for Dummies}
	\item \href{https://zhehaomao.com/blog/fpga/2014/05/24/sockit-10.html}{Interrupts} - Ein Beispiel wie man auf dem \ac{FPGA} Interrupts für den \ac{HPS} erzeugt und im Linux nutzt.
	\item \href{https://yurovsky.github.io/2014/10/10/linux-uio-gpio-interrupt/}{Wie man Linux Interrupts im Userspace verarbeitet}
 \end{itemize}