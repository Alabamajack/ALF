\chapter{Zusammenfassung}
Im Rahmen dieses HSP Projekts konnten die anfangs gesteckten Ziele weitgehend erreicht werden. Durch Ersetzen des Raspberry Pi durch eine flexible (FPGA) und leistungsstarke (ARM A9 Dualcore) Lösung, wurde eine Möglichkeit geschaffen sowohl Echtzeitbedingungen als auch low level IO Operationen (I2C, SPI, PWM etc.) von dem Hauptprozessoren zu entfernen. Durch diesen Schritt sind auf dem Hauptprozessor genug Leisutungsreserven frei um in zukünftigen Anwendungen Berechnungen, wie z.B. zur Berechnung von SLAM Algorithmen, direkt auf den Hauptprozessoren durchzuführen. Gleichzeitig hat das FPGA noch genug freie Logik zur Verfügung, um Teile von SLAM Algorithmen u.U. direkt auf Hardware rechnen zu können. Gleichzeitig konnten alle Funktionalitäten der Ausgangsbasis erhalten werden, wodurch dieses Projekt eine gute Grundlage für die Bearbeitung durch weitere Gruppen bildet.
